\section{Part II Results}
\label{sec:part2}

\paragraph{}
Graphs based on empirical and explanations for part 2 will be here.  We should
include a subsection on our mathematical analysis of the upper bound and
(hopefully) expected MST weight (could potentially plot these alongside the
empircal data).

\paragraph{}
Figure~\ref{fig:part2} shows empircal data is shown for all values of $|V|$
between $2$ to $8192$ (inclusive) which are powers of $1.5^n$ or $2^n$.  The
average MST weight per line for each of these 34 $|V|$ values is determined by
an average of at least 50 graphs.  In total, over $10,000$ graphs were generated
and analyzed to produce this plot; more details about how these graphs were
generated is available in Section~\ref{sec:tbd}.  We also computed $99.95\%$
confidence intervals of these averages but omit them from the figure because
they are so small.  The tiny size of these ranges allows us to say with high
confidence that our results here are statistically significant.

\begin{figure}[htb!]
\centering
\includegraphics[width=0.50\textwidth]{figures/part2.pdf}
\caption{Empircal data showing how the average MST weight increase with $|V|$
  for complete graphs.}
\label{fig:part2}
\end{figure}

\paragraph{}
Discuss how we did the best-fit analysis.  We used the Marquard-Levenberg
algorithm to fit a range of growth rates from constant to exponential.
Figure~\ref{fig:part2-fit} shows how each of these fits performs for each case.
The equations we fit follow (let $x = |V|$):
\begin{eqnarray*}
constant&:& c \\
log&:& m \cdot log_2(x) + c \\
polynomial / log&:& m \cdot \frac{x^p}{log_2(x)} + c \\
polynomial&:& m \cdot x^p + c \\
polynomial \cdot log&:& m \cdot x^p \cdot log_2(x) + c \\
exponential&:& m \cdot b^x + c
\end{eqnarray*}

\begin{figure*}[htb!]
\centering
\mbox{
\subfigure[Best-fit lines for random edge weights in the range $0$ to $1$ inclusive.
  ]{\label{fig:part2-fit-edge}\includegraphics[width=0.5\linewidth,angle=0]{figures/part2-fit-edge}}
\quad
\subfigure[Best-fit lines for random vertex positions in 2-D unit space.
  ]{\label{fig:part2-fit-loc2}\includegraphics[width=0.5\linewidth,angle=0]{figures/part2-fit-loc2}}
}
\mbox{
\subfigure[Best-fit lines for random vertex positions in 3-D unit space.
  ]{\label{fig:part2-fit-loc3}\includegraphics[width=0.5\linewidth,angle=0]{figures/part2-fit-loc3}}
\quad
\subfigure[Best-fit lines for random vertex positions in 4-D unit space.
  ]{\label{fig:part2-fit-loc4}\includegraphics[width=0.5\linewidth,angle=0]{figures/part2-fit-loc4}}
}
\label{fig:part2-fit}
\end{figure*}

\paragraph{}
Describe our best guesses for how each grows (show the best best-fit equations
with values filled in as on the plots).  Describe why we chose that fit.  Try to
explain what the heck is going on with the random edge weights one.  Make sure
to explain that poly/log, poly, and poly*log are all very good fits - they are
almost indistinguishable on the graph but that poly is best; also not that
const, log, and exp are clearly bad fits.
